
\section{Variational Gaussian Process}

Variational models introduce latent variables to the variational
family, providing a rich construction for posterior
approximation~\citep{ranganath2015hierarchical}.  Here we introduce
the \acrfull{VGP}, a Bayesian nonparametric variational model that is
based on the \acrlong{GP}.  The \gls{GP} provides a class of latent
variables that lets us capture downstream distributions with varying
complexity.

We first review variational models and Gaussian processes.  We then
outline the mechanics of the \gls{VGP} and prove that it is a
universal approximator.

\subsection{Variational models}
\label{sec:background}

Let $p(\mbz\g\mbx)$ denote a posterior distribution over $d$ latent
variables $\mbz=(z_1,\ldots,z_d)$ conditioned on a data set
$\mbx$. For a family of distributions $q(\mbz; \mblambda)$
parameterized by $\mblambda$, variational inference seeks to minimize
the divergence
$\operatorname{KL}(q(\mbz; \mblambda)\gg p(\mbz\g\mbx))$.  This is
equivalent to maximizing the
\gls{ELBO}~\citep{wainwright2008graphical}.  The \gls{ELBO} can be
written as a sum of the expected log likelihood of the data and the KL
divergence between the variational distribution and the prior,
\begin{equation}
\label{eq:bound_autoencoder}
\cL = \mathbb{E}_{q(\mbz; \mblambda)}[\log p(\mbx\mid\mbz)] -
\operatorname{KL}(q(\mbz; \mblambda)\|p(\mbz)).
\end{equation}
Traditionally, variational inference considers a tractable family of
distributions with analytic forms for its density. A common
specification is a fully factorized distribution
$\prod_i q(z_i; \lambda_i)$, also known as the mean-field
family. While mean-field families lead to efficient computation,
they limit the expressiveness of the approximation.

The variational family of distributions can be interpreted as a model
of the latent variables $\mbz$, and it can be made richer by
introducing new latent variables. Hierarchical variational models
consider distributions specified by a variational prior of the
mean-field parameters $q(\mblambda;\mbtheta)$ and a factorized
``likelihood'' $\prod_i q(z_i\mid\lambda_i)$.  This specifies the
variational model,
\begin{equation}
\label{eq:hvm}
q(\mbz;\mbtheta)=\int
\Big[\prod_i q(z_i\mid\lambda_i)\Big]
q(\mblambda;\mbtheta)\d\mblambda,
\end{equation}
which is governed by prior hyperparameters $\mbtheta$.  Hierarchical
variational models are richer than classical variational
families---their expressiveness is determined by the complexity
of the prior $q(\mblambda)$. Many expressive variational
approximations can be viewed under this
construct~\citep{saul1996exploiting,jaakola1998improving,rezende2015variational,tran2015copula}.

\subsection{Gaussian Processes}
\label{sec:background:gaussian}

We now review the
\glsreset{GP}\gls{GP}~\citep{rasmussen2006gaussian}. Consider a data
set of $m$ source-target pairs $\cD=\{(\mbs_n,\mbt_n)\}_{n=1}^m$,
where each source $\mbs_n$ has $c$ covariates paired with a
multi-dimensional target $\mbt_n\in\mathbb{R}^{d}$. We aim to learn a
function over all source-target pairs,
$
\mbt_n = f(\mbs_n),
$
where $f:\mathbb{R}^c\to\mathbb{R}^d$ is unknown. Let the function
$f$ decouple as $f=(f_1,\ldots,f_d)$, where each
$f_i:\mathbb{R}^c\to\mathbb{R}$. \gls{GP} regression estimates the
functional form of $f$ by placing a prior,
\begin{equation*}
p(f) = \prod_{i=1}^d \mathcal{GP}(f_i; \mbzero, \Kss),
\end{equation*}
where $\Kss$ denotes a covariance function $k(\mbs,\mbs')$ evaluated over pairs
of inputs $\mbs,\mbs'\in\mathbb{R}^c$.
In this paper, we consider \gls{ARD} kernels
\begin{equation}
k(\mbs,\mbs') = \sigma^2_{\textsc{ard}} \exp\Big(
-\frac{1}{2}\sum_{j=1}^c \omega_j(s_j - s'_j)^2
\Big),
\label{eq:ard}
\end{equation}
with parameters
$\mbtheta=(\sigma^2_{\textsc{ard}},\omega_1,\ldots,\omega_c)$. The
weights $\omega_j$ tune the importance of each dimension.
They can be driven to zero during inference, leading to automatic
dimensionality reduction.

Given data $\cD$, the conditional distribution of the \gls{GP} forms a
distribution over mappings which interpolate between input-output
pairs,
\begin{equation}
  p(f \g \cD) = \prod_{i=1}^d \mathcal{GP}(f_i; \Kxis \Kss^{-1} \mbt_i, \Kxixi -
  \Kxis \Kss^{-1} \Kxis^\top).
  \label{eq:gp_posterior}
\end{equation}
Here, $\Kxis$ denotes the covariance function $k(\mbxi, \mbs)$ for an
input $\mbxi$ and over all data inputs $\mbs_n$, and $\mbt_i$
represents the $i^{th}$ output dimension.

\subsection{Variational Gaussian Processes}
\label{sec:vgp}

We describe the \acrfull{VGP}, a Bayesian nonparametric variational
model that admits arbitrary structures to match posterior
distributions. The \gls{VGP} generates $\mbz$ by generating latent
inputs, warping them with random non-linear mappings, and using the
warped inputs as parameters to a mean-field distribution.  The random
mappings are drawn conditional on ``variational data,'' which are
variational parameters. We will show that the
\gls{VGP} enables samples from the mean-field to follow arbitrarily
complex posteriors.

\begin{figure}[tb]
\begin{subfigure}[t]{0.5\columnwidth}
  \centering
  \begin{tikzpicture}[x=1.7cm,y=1.8cm]

  % Nodes
  \node[obs]                  (z)      {$z_i$} ;
  \node[latent, left=of z]    (f)      {$f_i$} ;
  \node[latent, left=of f]    (xi)      {$\mbxi$} ;

  \factor[above=of f, xshift=-1.25cm] {theta} {$\mbtheta$} {} {};
  \factor[above=of f] {D} {\hspace{0.5cm}$\mathcal{D}=\{(\mbs,\mbt)\}$} {} {};
  %\factor[above=of f] {D} {$\mathcal{D}$} {} {};

  % Edges
  \edge{xi}{f};
  \edge{f}{z};
  \edge{theta}{f};
  \edge{D}{f};

  % Plates
  \plate[inner sep=0.35cm, yshift=0.15cm,
    label={[xshift=-14pt,yshift=14pt]south east:$d$}] {plate1} {
    (z)(f)
  } {};

\end{tikzpicture}

  \vspace{1ex}
  \caption{\textsc{variational model}}
  \label{sub:vgp}
\end{subfigure}
\begin{subfigure}[t]{0.5\columnwidth}
  \centering
  \begin{tikzpicture}[x=1.7cm,y=1.8cm]

  % Nodes
  \node[latent]               (z)      {$z_{i}$} ;
  \node[obs, right=of z]      (x)      {$\mbx$} ;

  % Edges
  \edge{z}{x};

  % Plates
  \plate[inner sep=0.35cm, yshift=0.15cm,
    label={[xshift=-14pt,yshift=14pt]south east:$d$}] {plate1} {
    (z)
  } {};
  %\plate[inner sep=0.35cm, yshift=0.15cm,
  %  label={[xshift=-14pt,yshift=14pt]south east:$N$}] {plate2} {
  %  (z)(x)(plate1)
  %} {};

\end{tikzpicture}

  \vspace{1ex}
  \caption{\textsc{generative model}}
  \label{sub:black_box}
\end{subfigure}
\caption{\textbf{(a)} Graphical model of the \acrlong{VGP}.
The \gls{VGP} generates samples of latent variables $\mbz$ by
evaluating random non-linear mappings of latent inputs $\mbxi$, and
then drawing mean-field samples parameterized by the mapping.  These
latent variables aim to follow the posterior distribution for a
generative model \textbf{(b)}, conditioned on data $\mbx$.
}
\label{fig:vgp}
\end{figure}


The \gls{VGP} specifies the following
generative process for posterior latent variables $\mbz$:
\begin{enumerate}
\item Draw latent input $\mbxi \in \mathbb{R}^c$:
$\mbxi\sim\mathcal{N}(\mbzero,\mbI).$
\item Draw non-linear mapping $f: \mathbb{R}^c\to\mathbb{R}^d$ conditioned on $\cD$:
$f\sim \prod_{i=1}^d\mathcal{GP}(\mbzero, \Kxixi) \g \cD.$
\item Draw approximate posterior samples $\mbz\in\operatorname{supp}(p)$:
$\mbz=(z_1,\ldots,z_d)\sim \prod_{i=1}^d
q(f_i(\mbxi) ).$
\end{enumerate}
\myfig{vgp} displays a graphical model for the \gls{VGP}.
Here, $\cD=\{(\mbs_n,\mbt_n)\}_{n=1}^m$ represents variational data, comprising
input-output pairs that are parameters to the variational
distribution.
Marginalizing over all latent inputs and non-linear mappings, the
\gls{VGP} is
\begin{equation}
\qvgp(\mbz; \mbtheta, \cD)
=
\iint
\left[
\prod_{i=1}^d q(z_i\g f_i(\mbxi))
\right]
\left[
\prod_{i=1}^d \mathcal{GP}(f_i; \mbzero, \Kxixi) \g \cD
\right]
\mathcal{N}(\mbxi; \mbzero,\mbI)
\d f \d\mbxi.
\label{eq:vgp}
\end{equation}
The \gls{VGP} is parameterized by kernel hyperparameters $\mbtheta$ and
variational data.

As a variational model, the \gls{VGP} forms an infinite ensemble of
mean-field distributions. A mean-field distribution is given in the
first term of the integrand above. It is
\emph{conditional} on a fixed function $f(\cdot)$ and input $\mbxi$;
the $d$ outputs $f_i(\mbxi)=\lambda_i$ are the mean-field's
parameters. The \gls{VGP} is a form of a hierarchical variational
model~(\myeqp{hvm})~\citep{ranganath2015hierarchical}. It places a continuous
Bayesian nonparametric prior over mean-field parameters.

Unlike the mean-field, the \gls{VGP} can capture correlation between the
latent variables. The reason is that it evaluates the $d$ independent
\gls{GP} draws at the same latent input $\mbxi$. This induces
correlation between their outputs, the mean-field parameters, and thus
also correlation between the latent variables. Further, the \gls{VGP} is
flexible. The complex non-linear mappings drawn from the \gls{GP}
allow it to capture complex discrete and continuous posteriors.

We emphasize that the \gls{VGP} needs variational data. Unlike typical
\gls{GP} regression, there are no observed data available to learn a
distribution over non-linear mappings of the latent variables $\mbz$.
Thus the "data" are variational parameters that appear in the
conditional distribution of $f$ in \myeqp{gp_posterior}. They anchor
the random non-linear mappings at certain input-ouput pairs. When
optimizing the \gls{VGP}, the learned variational data enables finds a
distribution of the latent variables that closely follows the
posterior.

\subsection{Universal approximation theorem}
\label{sec:vgp:universal}



To understand the capacity of the \gls{VGP} for representing complex
posterior distributions,
we analyze the role of the \acrlong{GP}. For simplicity,
suppose the latent variables $\mbz$ are real-valued, and the \gls{VGP}
treats the output of the function draws from the \gls{GP} as posterior
samples.
Consider the optimal
function $f^*$, which is the transformation such that when we draw
$\mbxi\sim\mathcal{N}(\mbzero,\mbI)$ and calculate $\mbz= f^*(\mbxi)$,
the resulting distribution of $\mbz$ \emph{is} the posterior distribution.

An explicit construction of $f^*$ exists if the dimension of the
latent input $\mbxi$ is equal to the number of latent variables. Let
$P^{-1}$ denote the inverse posterior CDF and $\Phi$ the standard
normal CDF. Using techniques common in copula
literature~\citep{nelsen2006introduction}, the optimal function is
\begin{equation*}
f^*(\mbxi) = P^{-1}(\Phi(\xi_1),\ldots,\Phi(\xi_d)).
\end{equation*}
Imagine generating samples $\mbz$ using this function. For latent
input $\mbxi\sim\mathcal{N}(\mbzero,\mbI)$, the standard normal CDF
$\Phi$ applies the probability integral transform: it squashes
$\xi_i$ such that its output $u_i=\Phi(\xi_i)$ is uniformly
distributed on $[0,1]$. The inverse posterior CDF then transforms the
uniform random variables $P^{-1}(u_1,\ldots,u_d)=\mbz$ to follow
the posterior. The function produces exact posterior samples.

In the \gls{VGP}, the random function interpolates the values in the variational data,
which are optimized to minimize the KL divergence.
Thus, during inference, the distribution of the \gls{GP} learns to concentrate around this
optimal function. This perspective provides intuition
behind the following result.


\newcommand{\limittheorem}{
Let $q(\mbz;\mbtheta, \cD)$ denote the \acrlong{VGP}. Consider a
posterior distribution $p(\mbz\g\mbx)$ with a finite number of latent
variables and continuous quantile function (inverse CDF). There exists
a sequence of parameters
$(\mbtheta_k, \cD_k)$ such that
\begin{equation*}
\lim_{k \to \infty} \operatorname{KL}(q(\mbz; \mbtheta_k, \cD_k) \gg p(\mbz\g\mbx))
= 0
.
\end{equation*}
}
\begin{theorem}
[Universal approximation]
\label{theorem:limit}
\limittheorem
\end{theorem}
See \myappendix{zero} for a proof. \mytheorem{limit} states that any
posterior distribution with strictly positive density
can be represented by a \gls{VGP}.
Thus the \gls{VGP} is a
flexible model for learning posterior distributions.





